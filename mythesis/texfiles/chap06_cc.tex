\chapter{总结与展望}

本文详细讲述了作者在博士期间的主要研究工作。作者在读博期间进入超冷原子物理这个方兴未艾的领域,并对其中的超冷原子光晶格体系进行了一系列深入的研究,研究重点集中在其拓扑性质和含时动力学性质方面。本文详细论述了作者在这两方面的几个工作,下面分别进行总结。


\section{拓扑}

在拓扑方面:

1)作者和合作者们研究了光晶格中的拓扑电子输运问题,提出了 Creutz 梯上的拓扑电子泵模型,给出了该模型完整的拓扑相图,并提出了该模型中以偶数为内禀输运单位的绝热拓扑电荷输运过程;我们发现,在该模型中的各种拓扑电子输运路径和其相图上不同部分的 Zak 相角之间有密切关系;我们还通过数值模拟的方法对该模型的绝热拓扑输运过程进行了验证。
(见第 \ref{chap:chargepump} 章)

2)作者和合作者们使用了深度学习的方法,利用深层人工神经网络进行了拓扑物态的机器学习;我们对机器进行一维 AIII 类 和 二维 A 类拓扑物态的识别训练,利用绕数和Chern数作为训练标签,使之取得了超过90\%的预言精度,且结果具有推广性,能够以较高精度预言未见过的类别;我们打开深层神经网络,发现在其中间隐藏层提取到了全局的 Berry 曲率场的信息——也就是机器并不止学会了一个拓扑数,它通过学习全局知识学习到了拓扑物态的精髓,因此“学会”了分类拓扑相。
(见第 \ref{sec:topoml} 章)

3)作者为人们常用到的二维材料的陈数算法\cite{chern2005}开发了专门的基于 python3 语言 和 基于 Wolfram Mathematica 语言的程序包,并在本论文中证明了该算法的有效性与量子化、规范不变等性质(见附录 \ref{sec:chern})。





\section{动力学}

在含时动力学方面:

1)作者和合作者们证明了一类有 Hubbard 相互作用的体系中受单粒子对称性保护的动力学对称性,该定理能够将几个截然不同的实验现象\cite{hubbard-expan-2010,hubbard-expan-2012,mbl1d,twobody-2017}联系起来;作者和合作者们用严格对角化的数值方法对三类例子进行了检验;这三类例子分别联系着 Fermi Hubbard 扩散实验\inlinecite{hubbard-expan-2010,hubbard-expan-2012},玻色子带磁通的 Hubbard 模型上少体相互作用极限的实验\inlinecite{twobody-2017},和一维准无序多体局域化的实验\inlinecite{mbl1d}。
(见第\ref{chap:dynm} 章 \ref{sec:dynsymm} 节)

2)作者和合作者们证明了在二分 Fermi Hubbard 模型中电荷和自旋输运方面存在精确关系;作者在本论文中还讨论了该定理的推广、以及温度效应等。
(见第\ref{chap:dynm} 章 \ref{sec:diffusion})

3)作者和合作者们对周期含时驱动的 Floquet 系统进行了研究,并重点研究了
在近共振高频驱动的框架下的正方晶格 Fermi Hubbard 模型,该模型在蜂巢晶格中的实现在2018年由苏黎世联邦理工学院的实验小组报告\cite{correlated-tunnel-expr-2018-shaking}。利用标准路径积分平均场的方法,我们计算出了完整的平均场相图,并发现了在小的有效相互作用区域存在的铁磁相和相分离;这种铁磁相出现的机制完全是由于近共振的高频驱动以及由其引起的关联隧穿效应,有别于凝聚态中周知的 Stoner 机制。这种新奇的铁磁相有望在冷原子实验中被检验。
(见第\ref{chap:chargepump} 章)



\section{展望}

自从上世纪人类利用激光冷却和蒸发冷却的方法将稀薄的原子气体云冷却到百纳开尔文级别的低温,以及利用这样的低温实现了碱金属原子的玻色爱因斯坦凝聚\cite{bec1995a,bec1995b},人们对稀薄的超冷原子气体系统展开了一系列来自理论和实验的研究。近二十年时间以来,超冷原子物理领域的发展迅猛,研究成果愈加丰富。一方面,超冷原子气体体系有自己内在的丰富物理,例如少体物理问题,另一方面,人们利用超冷原子气体系统进行了一系列广泛关注的量子多体问题的研究,例如对一系列量子霍尔效应同源的物理系统与过程的研究\cite{topo2016zoller,harper1,harper2,zak-expr-2013,chern-expr-2015,ab-expr-2015,wilsonline-expr-2016,haldane-expr-2014,charge-pump-expr-2016-de,charge-pump-expr-2016-jp,4dqhall-expr-2018},对 Hubbard 模型的量子模拟\cite{hubbard-expan-2010,hubbard-expan-2012,microscope5,microscope6,af1,af2,af3,canted,incommensurate,af_long_range,pair_attractive,hidden_af_doped,charge-diffusion,spin-diffusion,floq-hubb-expr-2018,correlated-tunnel-expr-2018-shaking,correlated-tunnel-expr-2018-raman},对本征态热化假说\cite{thermalize-entropy-2016}与多体局域化\cite{mbl1d,mbl2d}等复杂问题的研究等。由于超冷原子系统的干净、可控,人们还可以研究一些传统凝聚态难以实现的系统与过程,例如周期驱动的体系\cite{haldane-expr-2014,floq-hubb-expr-2018,correlated-tunnel-expr-2018-shaking}与动力学淬灭过程。而利用诸如 Feshbach 共振等手段,体系的参数可以被在很大的范围内调控,这也大大拓宽了冷原子物理可以涉及的范围。

近来,超冷原子物理领域在探测手段和调控技术等方面取得了许多重要进展,例如量子气体显微镜的发展,这些进展使人们可以对强相互作用强关联体系的物理做更深入的研究与探测,例如单格点分辨、时间分辨的探测。人们已经利用这些喜人的成果实现了诸如一维 Luttinger 液体\cite{hidden_af_doped}、斜面反铁磁序\cite{canted}等物理。而这些不断的进展为超冷原子物理领域提供了丰富的方法和手段,和许多本身就丰富的物理。利用这些成果,人们未来有希望对许多强关联体系建立更深入的理解,有希望对更多诸如动力学淬灭的体系展开研究。例如作者博士期间和合作者们提出的近共振驱动的铁磁相\cite{floqhubb},有希望在未来的实验中加以验证。而超冷原子光晶格体系作为本领域尤为重要的领域和平台,未来有希望排上更大用场,甚至在诸如量子计算和量子信息方面发挥作用\cite{bloch2012}。

未来可期。