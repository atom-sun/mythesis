\thusetup{
  %******************************
  % 注意:
  %   1. 配置里面不要出现空行
  %   2. 不需要的配置信息可以删除
  %******************************
  %
  %=====
  % 秘级
  %=====
  secretlevel={秘密},
  secretyear={10},
  %
  %=========
  % 中文信息
  %=========
  ctitle={超冷原子光晶格系统的\\拓扑和动力学研究},
  cdegree={理学博士},
  cdepartment={高等研究院},
  cmajor={物理学},
  cauthor={孙宁},
  csupervisor={翟荟教授},
  % cassosupervisor={陈文光教授}, % 副指导老师
  % ccosupervisor={某某某教授}, % 联合指导老师
  % 日期自动使用当前时间,若需指定按如下方式修改:
  % cdate={超新星纪元},
  %
  % 博士后专有部分
  cfirstdiscipline={物理学},
  cseconddiscipline={系统结构},
  postdoctordate={2009年7月——2011年7月},
  id={编号}, % 可以留空: id={},
  udc={UDC}, % 可以留空
  catalognumber={分类号}, % 可以留空
  %
  %=========
  % 英文信息
  %=========
  etitle={Topology and Dynamics of Ultracold Atoms in Optical Lattices},
  % 这块比较复杂,需要分情况讨论:
  % 1. 学术型硕士
  %    edegree:必须为Master of Arts或Master of Science(注意大小写)
  %             “哲学、文学、历史学、法学、教育学、艺术学门类,公共管理学科
  %              填写Master of Arts,其它填写Master of Science”
  %    emajor:“获得一级学科授权的学科填写一级学科名称,其它填写二级学科名称”
  % 2. 专业型硕士
  %    edegree:“填写专业学位英文名称全称”
  %    emajor:“工程硕士填写工程领域,其它专业学位不填写此项”
  % 3. 学术型博士
  %    edegree:Doctor of Philosophy(注意大小写)
  %    emajor:“获得一级学科授权的学科填写一级学科名称,其它填写二级学科名称”
  % 4. 专业型博士
  %    edegree:“填写专业学位英文名称全称”
  %    emajor:不填写此项
  edegree={Doctor of Philosophy},
  emajor={Physics},
  eauthor={Ning Sun},
  esupervisor={Professor Hui Zhai},
  % eassosupervisor={Chen Wenguang},
  % 日期自动生成,若需指定按如下方式修改:
  % edate={December, 2005}
  %
  % 关键词用“英文逗号”分割
  ckeywords={冷原子, 光晶格, 拓扑, 动力学, 机器学习},
  ekeywords={cold atom, optical lattice, topological matter, dynamics, machine learning}
}

% 定义中英文摘要和关键字
\begin{cabstract}
% 超冷原子物理对于进行凝聚态多体系统的量子模拟、量子计算与量子信息、量子精密调控测量等方面的研究来说具有重要价值,这其中,超冷原子光晶格体系又是进行量子多体研究的重要平台。自2002年人类首次报告在超冷原子光晶格体系中观测到 Mott-超流相变,超冷原子光晶格体系在过去的十几年里受到了理论和实验方面的广泛研究。
% 近年来,得益于调控技术与探测手段方面的进步,人们可以利用光晶格超冷原子气体系统进行更丰富的研究。例如,过去十年里量子气体显微镜的发展,使得人们能够对强相互作用体系的演化进行单原子层面的深入研究,在态的制备等方面也具有更大自由度,可以研究周期含时驱动与动力学淬灭的过程。
超冷原子光晶格系统是进行量子多体物理研究的重要平台,在过去十多年里,在这个领域有很多重要进展。探测手段与调控技术的进步使人们可以利用光晶格来进行越来越丰富的研究,例如,近年来量子气体显微镜的发展,使人们能够对一些强相互作用体系进行单格点分辨、时间分辨的探测与研究,在态的制备方面也具有更大的自由度。此外,光晶格体系还可以用来进行周期含时驱动与动力学淬灭等过程的研究。

对超冷原子光晶格体系的理论研究是作者博士期间的主要研究方向,研究重点是其拓扑和动力学性质。本文围绕这两方面介绍了作者博士期间的主要工作。

在拓扑方面:
1)作者和合作者们研究了光晶格中的拓扑电子输运问题,提出了 Creutz 梯上的拓扑电子泵模型,给出了该模型完整的拓扑相图,并发现了模型中存在以偶数为内禀输运单位的绝热拓扑输运路径;
2)作者和合作者们利用深度学习方法研究了拓扑物态的机器学习问题,对机器进行一维 AIII 类 和 二维 A 类的拓扑物态的识别训练,使之取得了超过90\%的精度,且结果具有推广性,能够以较高精度预言未见过的类别;
3)作者为人们常用到的二维材料的陈数算法%\cite{chern2005}
开发了专门的程序包,并证明了该算法的有效性与量子化、规范不变等性质(见附录)。

在含时动力学方面:
1)作者和合作者们证明了一类有 Hubbard 相互作用的体系中受单粒子对称性保护的动力学对称性,该定理能够将几个截然不同的实验现象
%\cite{hubbard-expan-2010,hubbard-expan-2012,mbl1d,twobody-2017}
联系在一起;作者和合作者们用严格对角化的方法对三类例子进行了数值检验;
2)作者和合作者们证明了在二分 Fermi Hubbard 模型中电荷和自旋输运方面存在精确关系;
3)作者和合作者们对周期含时驱动的 Floquet 系统进行了研究,并重点研究了
在近共振高频驱动的框架下的正方晶格 Fermi Hubbard 模型,利用标准路径积分平均场的方法,计算出了完整的平均场相图,并发现了在小的有效相互作用区域存在的铁磁相和相分离;这种铁磁相出现的机制完全是由于近共振的高频驱动以及由其引起的关联隧穿效应,有别于凝聚态中周知的 Stoner 机制。
\end{cabstract}

% 如果习惯关键字跟在摘要文字后面,可以用直接命令来设置,如下:
% ckeywords={冷原子, 光晶格, 拓扑, 动力学, 周期驱动,机器学习},

\begin{eabstract}
Ultracold Atoms are valuable platform for quantum simulating condensed matter many-body physics, and for the study of quantum computation and quantum information, etc. Among all ultracold atom systems, the ultracold atoms in optical lattices are especially precious for the study of quantum many-body physics. Since the discovery of Mott-SF transition in optical lattice in 2002, the study of ultracold atoms in optical lattices are rich from both the theoretical aspect and the experimental aspect. Recently, it profits even more from the improvement in detection and manipulation techniques. For example, the development of quantum gas microscope has enable people to study quantum many-body system more deeply and precisely. It also offers more degrees of freedom in preparing states. Hence, the study of quench dynamics and periodic driven systems are available in ultracold atom optical lattices.

In the pursuing of doctorate degree, the author has been spending a lot of time in the theoretical study of such ultracold atom optical lattice systems. The focus is on the topological aspect and the dynamics. In this thesis, the work on these two fields are discussed in details. 

In the topological aspect:
1) The author and collaborators study the topological charge transport in optical lattices. We propose the generalized Creutz model for the topological charge pumping, with the complete topological phase diagram given. And a novel type charge pump has been discovered in this model where only even units of particles can be transported in one cycle.
2) The author and collaborators study the machine learning topological matter problems using deep learning method. We train the machine to recognize different topological phases in 1D AIII class and 2D A class and D class, with the accuracy over 90\% after training. The machine also has a generalizing ability after training in the sense that it can recognize phases that has not been seen.
% 3) For the common algorithm for calculating Chern number, the author develop special libraries of codes. The author also proves some theorems and properties demonstrating it's efficiency and powerfulness. (See appendix)

In the dynamics aspect:
1) The author and collaborators propose a symmetry protected dynamical symmetry in a series of Hubbard models, and prove the corresponding theorem. This theorem provides a unified way of understanding to several quite different experiments recently in cold atom community. We also do numerics examination using exact diagonalization for three classes of examples.
2) The author and collaborators propose some exact relations in the observation of charge transport and spin transport in bipartite Fermi Hubbard model, and prove the corresponding theorem.
3) The author and collaborators study the periodically driven optical lattices, with a deep focus on the nearly resonant driven Fermi Hubbard model on square lattice. We use the standard path integral approach to carry out a mean-field treatment to the system, and numerically calculate the complete mean-filed phase diagram. We find that in small effective interaction regime, there are a ferromagnetism (FM) phase and phase separation phase. This emergent FM phase is novel in that it is induced wholly by the correlated tunneling effect which is actually generated by the nearly resonant high-frequency driving. This is quite different from the well-known Stoner mechanism. 
\end{eabstract}

% ekeywords={cold atom, optical lattice, topological matter, dynamics, Floquet, machine learning}
