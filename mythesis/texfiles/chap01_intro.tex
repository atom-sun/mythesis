\chapter{超冷原子光晶格系统简介}
\label{cha:intro}

1995年,人类首次在稀薄的冷原子气体中直接观测到碱金属原子的玻色爱因斯坦凝聚\cite{bec1995cornell,bec1995ketterle}。自那时起,冷原子物理作为一个新兴的研究领域,在之后的20多年时间里蓬勃发展。人们对稀薄的超冷原子气体系统展开研究,取得了许多重要的成果,这些成果揭示了冷原子物理在进行凝聚态多体系统的量子模拟、量子计算与量子信息、量子体系含时动力学研究、量子光学、以及量子器件与精密测量等方面研究的重要价值\cite{bloch2012}。例如,通过激光驻波作用在超冷原子气体上,形成周期性势场(光晶格),可以用来模拟凝聚态物理中的 Hubbard 模型等许多模型在不同参数条件下的行为。冷原子物理所做的量子模拟并非简单的复制和模仿,而是创造性的模拟。一方面,经过与现有最前沿量子蒙特卡洛等方法的结果进行校准,利用光晶格中的超冷原子系统实现了对具体模型的高精度的仿真;另一方面,经过校准后的光晶格模拟器又可以模拟更极端参数下的行为,和更大的体系,而经典计算机对很大的体系缺乏有效的数值模拟算法,从而实现量子模拟对经典计算机的超越\cite{feynman1982,zoller2012goals}。近年来,得益于调控手段与探测技术的双方面的进步,人们可以对强相互作用体系的演化进行单原子层面的深入研究,甚至研究一些传统凝聚态无法实现的体系,如含时驱动体系和非平衡态过程。超冷原子光晶格体系的研究价值逐渐彰显。对这种光晶格中超冷原子气体的理论研究是作者博士期间的主要研究方向,研究重点是其拓扑和含时动力学性质。下面将做具体介绍。


\section{冷原子光晶格体系与量子模拟}

在凝聚态物理中,固体中的电子运动在带电离子实规则排列形成周期性的晶格势场中。对这样的周期性结构,冷原子物理中可以用光晶格来进行模拟。对一团中性超冷原子气体,利用激光驻波作用其上可以形成周期性的光晶格势场,用来模拟固体材料中离子实形成的周期性势场,而超冷原子气体在势场中的运动行为则用来模拟固体中的电子的运动。通过将几个方向的激光进行不同角度的叠加,可以产生出正方晶格、三角晶格、六角晶格、蜂巢晶格、甚至 Kagome 晶格等,可以用来模拟许多凝聚态中的不同模型,包括带阻错的模型。光晶格中原子和原子间的相互作用一方面可以通过 Feshbach 共振的技术来进行调节\cite{feshbach2010},通过直接调节原子间的散射长度来实现格点上原子间的相互作用强度的调节。另一方面,也可以通过调节光晶格的势阱深度来调节原子间的相互作用与格点间跃迁动能之间的相对比($U/J$)\cite{bloch2012}。当光势阱较浅时,晶格格点上的局域化 Wannier 波函数有较宽的展宽,临近格点的瓦尼儿波函数有较大重合,跃迁矩阵元较大,相应的,$U/J$较小;而当势阱深度很深时,晶格格点上的 Wannier 波函数被更多的局域化在自己的格点上,与其他格点上的 Wannier 波函数重叠很小,相应的,跃迁矩阵元很小,体系的能带(与相互作用能量相比)趋近于平带,此时$U/J$则很大。例如,通过类似的方式,人们实现了三维光晶格中超冷原子气体从超流到 Mott 绝缘体的量子相变\cite{mott-sf-2002},以及许多不同的凝聚态体系的量子模拟。

过去,利用稀薄的超冷原子气体进行量子多体强关联现象的研究有过许多的理论和实验的进展\cite{bloch2008}。这些研究所关注的问题超出一般的弱的相互作用体系的描述,而是更集中于原子间强的相互作用所引起的效应,例如光晶格中的 Mott-Hubbard 相变,一维和二维体系中的强相互作用气体,以及在快速旋转的准二维气体中的最低朗道能级等。光晶格中的强关联费米气体也是一个重要的研究方向,这样的系统往往可以用来对传统凝聚态中的重要模型进行模拟,如 Hubbard 模型和 Heisenberg 自旋模型。研究这些重要的模型,以及这些体系中的铁磁/反铁磁关联、热力学性质、动力学性质以及输运性质的研究,对于理解传统凝聚态强关联系统的现象,例如分数量子霍尔效应和 高温超导等现象,有着重要帮助\cite{nagaosa}。

近来,冷原子实验在调控与探测光晶格中的中性原子技术方面有许多重大进展。例如,通过调节一系列参数,诸如散射长度、晶格势阱深度、外部禁闭等,人们可以绝热地调节中性原子气体间从很弱的相互作用到很强的相互作用。再如,量子气体显微镜的发展是过去十年中冷原子物理领域的重要进展\cite{},一方面,它允许人们对光晶格中的冷原子体系做晶格上的单原子探测,另一方面,它使得人们在态的制备方面有了更大的自由度,可以制备实空间密度算符的本征态(例如电荷密度波态和自选密度波态)作为初态进行研究。这些进展使得超冷原子光晶格系统对于进行量子多体强关联体系的研究来说更加具有价值。

可以利用超冷原子光晶格系统进行研究的量子多体系统包括\cite{ol2012}:

\begin{itemize}

\item 一大类 Hubbard 模型,带有格点上相互作用的单带模型乃至 $t-J$ 模型,包括费米子的,玻色子的,带磁通的,吸引相互作用的,排斥相互作用的,半填充的,掺杂的,磁中性的,有磁梯度的,等等,以及在测量到的长程序

\item 各种自旋模型,包括 Ising 模型,Heisenberg 模型,XXZ 模型等

\item 无序系统,包括 Anderson 局域化和多体局域化等系统,利用不公度的晶格实现准无序系统的 准一维 Aubry-Andre 模型,二维多体局域化模型等

\item 一大类量子霍尔效应(Quantum Hall)同源的物理,拓扑物态,(非交换)几何物理,如各种非平庸的 Berry 相,Zak 相,非阿贝尔 Wilson 路径等等,以及 Haldane 模型\cite{esslinger2014haldane},Harper-Hofstadter 模型\cite{hofstadter2013bloch,hofstadter2013ketterle},Thouless 拓扑电子泵,等等。

\item 不同维度的物理,包括一维 Luttinger 液体,自旋电荷分离,非公度的自旋波和电荷波激发,二维的短程关联和准长程序,三维的长程序等等。


光诱导的人工规范场\cite{lightgauge2014},晶格方块上的磁通等等。

\end{itemize}

等等。

此外,利用超冷原子光晶格体系还可以实现一些传统凝聚态很难实现的系统与过程,如周期含时驱动的系统和动力学淬灭过程。一方面,周期含时驱动的体系给人们 提供了生成一些有效稳态模型的方式,例如通过高频率周期含时驱动所生成的 Floquet-拓扑绝缘体\cite{floq-ti-2011};另一方面,与静态系统相比,周期含时驱动本身就会使系统的内在性质有很大不同,例如,对周期驱动体系的拓扑分类与静态系统有着内禀的不同\cite{rudner2013anomalous}。此外,通过对这些系统的研究,人们可以研究一些传统凝聚态很少研究或很难用实验研究的物理过程,例如非平衡态的问题,还有从单个纯态出发的动力学淬灭的过程,甚至包括本征态热化假说(Eigenstate Themalization Hypothesis)\cite{eth-entropy-2016}这样的量子力学基本问题等。




\section{晶格体系与拓扑物态}


凝聚态物理中一个重要的主题是对不同的物质的相的刻画与分类。通常来讲,这可以通过朗道的相变理论来理解,即物质的相由体系的对称性以及态的对称性自发破缺来刻画,态的对称性自发破缺形成物质的序。然而,从上世纪80年代开始,一系列对量子霍尔效应的研究打开了新世界的大门,有一类物态并没有破缺体系的对称性,但却可以定义某种序,这些序刻画着体系最基本的一些性质,而且对于材料的参数变化具有鲁棒性。这样的序可以用某个拓扑不变量来刻画,被称为体系的拓扑数,这类物态被称为拓扑物态。体系的某些基本性质,例如量子化的霍尔电导平台,都由这个拓扑数来刻画。材料的连续的微小的扰动不会改变体系的拓扑数,相应的也不会改变体系的那些基本的性质,只有发生较大变化,使体系经历量子相变(关闭能隙再打开),态的拓扑数才有可能改变。

拓扑物态中有一类拓扑绝缘体在过去十几年中被广泛研究\cite{topo2010}。拓扑绝缘体是指一类电子材料,它们的体材料如同普通的能带绝缘体一样,电子填充价带而导带空置,价带和导带之间有能隙,因此体材料并不导电,但其边界或表面上却局域着导电态的边缘态。这种边缘态受材料的拓扑数所保护,对于材料的微小、连续的扰动具有鲁棒性,并且对于体系的微小的无序性具有鲁棒性,不会与其他态发生背散射,这也正是其具有量子化的霍尔电导平台的原因。

拓扑物态所具有的良好性质吸引人们进行了许多理论和实验的研究,包括对拓扑材料的探索。而拓扑绝缘体之能发生于周期性晶格结构,也使得利用超冷原子光晶格系统进行探索成为可能。然而超冷原子系统进行研究的一个问题是,如何模拟电磁规范场。由于光晶格中的中性原子并不带电,并不能通过直接加电场或磁场的方式来使体系产生电磁规范场。对此,人们可以用光诱导产生人工规范场\cite{lightgauge2014}来进行研究。通过拉曼光耦合原子不同的内态,可以实现有效的自旋轨道耦合。类似地,通过拉曼光的帮助,利用AB效应的原理,可以实现晶格方块上的磁通\cite{hofstadter2013bloch,hofstadter2013ketterle}。此外,人们还可以对光晶格系统进行周期性含时驱动,使频闪意义上的有效静态模型具有非平庸的拓扑。用这样的方式,人们在冷原子系统中实现了 Haldane 模型\cite{esslinger2014haldane},一个典型的 Chern 类绝缘体,还有前面提到的 Floquet-拓扑绝缘体\cite{floq-ti-2011}。类似的,人们还实现了玻色子带磁通的的晶格模型\cite{bosonflux2017harvard}。

还有一些拓扑物态相关的现象和过程,如 Thouless 式的量子化绝热电子输运\cite{thouless1983},也在超冷原子光晶格系统中得到实现\cite{charge-pump-expr-2016-de,charge-pump-expr-2016-jp}。


\section{光晶格体系的动力学性质}

例如,二维 Fermi-Hubbard 模型的反常扩散,电子、自旋输运性质等。动力学。


\section{周期含时驱动的光晶格体系}
一类很有意思的系统是含时驱动的系统,这样的系统过去很难在传统的凝聚态系统中实现。现在得益于技术进步,凝聚态系统中已经有了一些周期含时驱动的系统实现出来,例如利用



\section{冷原子物理与机器学习算法}
机器学习是一门古老的学科。早在300年前,线性回归,高斯最小二乘法…… 早在100年前,pca,农业,种玉米……



作为机器学习中一个重要的分支,模式识别在过去几年取得了重要的突破。


