\chapter{超冷原子光晶格系统简介}
\label{cha:intro}

1995年,人类首次在稀薄的冷原子气体中直接观测到碱金属原子的玻色爱因斯坦凝聚\cite{bec1995a,bec1995b}。自那时起,冷原子物理作为一个新兴的研究领域,在之后的20多年时间里蓬勃发展。人们对稀薄的超冷原子气体系统展开研究,取得了许多重要的成果,这些成果揭示了冷原子物理在进行凝聚态多体系统的量子模拟、量子计算与量子信息、量子体系含时动力学研究、量子光学、以及量子器件与精密测量等方面研究的重要价值\cite{bloch2012}。例如,通过激光驻波作用在超冷原子气体上,形成周期性势场(光晶格),可以用来模拟凝聚态物理中的 Hubbard 模型等许多模型在不同参数条件下的行为。冷原子物理所做的量子模拟并非简单的复制和模仿,而是创造性的模拟。一方面,经过与现有最前沿量子蒙特卡洛等方法的结果进行校准,利用光晶格中的超冷原子系统实现了对具体模型的高精度的仿真;另一方面,经过校准后的光晶格模拟器又可以模拟更极端参数下的行为,和更大的体系,而经典计算机对很大的体系缺乏有效的数值模拟算法,从而实现量子模拟对经典计算机的超越\cite{feynman1982,qsgoals2012}。近年来,得益于调控手段与探测技术的双方面的进步,人们可以对强相互作用体系的演化进行单原子层面的深入研究,甚至研究一些传统凝聚态无法实现的体系,如含时驱动体系和非平衡态过程。超冷原子光晶格体系的研究价值逐渐彰显。对这种光晶格中超冷原子气体的理论研究是作者博士期间的主要研究方向,研究重点是其拓扑和含时动力学性质。下面将做具体介绍。


\section{超冷原子光晶格系统与量子模拟}

在凝聚态物理中,固体中的电子运动在带电离子实规则排列形成周期性的晶格势场中。对这样的周期性结构,冷原子物理中可以用光晶格来进行模拟。对一团中性超冷原子气体,利用激光驻波作用其上可以形成周期性的光晶格势场,用来模拟固体材料中离子实形成的周期性势场,而超冷原子气体在势场中的运动行为则用来模拟固体中的电子的运动行为。通过将几个方向的激光进行不同角度的叠加,可以产生出正方晶格、三角晶格、六角晶格、蜂巢晶格、甚至 Kagome 晶格等,可以用来模拟凝聚态中许多不同的模型,包括带阻错的模型。光晶格中原子和原子间的相互作用一方面可以通过 Feshbach 共振的技术来进行调节\cite{feshbach2010},通过直接调节原子间的散射长度来实现格点上原子间的相互作用强度的调节。另一方面,也可以通过调节光晶格的势阱深度来调节原子间的相互作用与格点间跃迁动能之间的相对大小($U/J$)\cite{bloch2012}。当光势阱较浅时,晶格格点上的局域化 Wannier 波函数有较宽的展宽,临近格点的 Wannier 波函数有较大重合,跃迁矩阵元较大,相应的,$U/J$较小;而当势阱深度很深时,晶格格点上的 Wannier 波函数被更多的局域化在自己的格点上,与其他格点上的 Wannier 波函数重叠很小,相应的,跃迁矩阵元很小,体系的能带(与相互作用能量相比)趋近于平带,此时$U/J$则很大。通过类似的方式,人们实现了三维光晶格中超冷原子气体从超流到 Mott 绝缘体的量子相变\cite{mott-sf-2002},以及许多不同的凝聚态体系的量子模拟。

过去,在利用稀薄的超冷原子气体进行量子多体强关联现象的研究方面有过许多的理论和实验的进展\cite{bloch2008}。这些研究所关注的问题超出一般的弱的相互作用体系的描述,而是更集中于原子间强的相互作用所引起的效应,例如光晶格中的 Mott-Hubbard 相变,一维和二维体系中的强相互作用气体,以及在快速旋转的准二维气体中的最低朗道能级等。光晶格中的强关联费米气体也是一个重要的研究方向,这样的系统往往可以用来对传统凝聚态中的重要模型进行模拟,如 Hubbard 模型和 Heisenberg 自旋模型。研究这些重要的模型,以及这些体系中的铁磁/反铁磁关联、热力学性质、动力学性质以及输运性质的研究,对于理解传统凝聚态强关联系统的现象,例如分数量子霍尔效应和 高温超导等现象,有着重要帮助\cite{nagaosa}。

近来,冷原子实验在调控与探测光晶格中的中性原子技术方面有许多重大进展。例如,通过调节一系列参数,诸如散射长度、晶格势阱深度、外部禁闭等,人们可以绝热地调节中性原子气体间从很弱的相互作用到很强的相互作用。再如,量子气体显微镜的发展是过去十年中冷原子物理领域的重要进展\cite{microscope1,microscope2,microscope3,microscope4,microscope5,microscope6},一方面,它允许人们对光晶格中的冷原子体系做晶格上的单原子探测,另一方面,它使得人们在态的制备方面有了更大的自由度,可以制备实空间密度算符的本征态(例如电荷密度波态和自选密度波态)作为初态进行研究。这些进展使得超冷原子光晶格系统对于进行量子多体强关联体系的研究来说更加具有价值。

可以利用超冷原子光晶格系统进行研究的量子多体系统包括\cite{olbook}:

\begin{itemize}

\item 一大类 Hubbard 模型\cite{hubbard-expan-2010,hubbard-expan-2012,microscope5,microscope6,af1,af2,af3,canted,incommensurate,af_long_range,pair_attractive,hidden_af_doped,charge-diffusion,spin-diffusion,floq-hubb-expr-2018,correlated-tunnel-expr-2018-shaking,correlated-tunnel-expr-2018-raman},带有格点上相互作用的单带模型乃至 $t-J$ 模型\cite{olbook},包括费米子的,玻色子的\cite{twobody-2017},带磁通的\cite{twobody-2017},吸引相互作用的\cite{pair_attractive},排斥相互作用的,半填充的\cite{microscope5,microscope6},掺杂的\cite{hidden_af_doped},磁中性的,有磁梯度的\cite{twobody-2017},有关联隧穿效应的\cite{floq-hubb-expr-2018,correlated-tunnel-expr-2018-shaking,correlated-tunnel-expr-2018-raman},等等,以及对其中的短程关联\cite{af1,af2,af3}、(准)长程关联\cite{af_long_range,canted,pair_attractive,hidden_af_doped}、输运性质\cite{charge-diffusion,spin-diffusion}、少体相互作用性质\cite{twobody-2017}等等的研究和测量。

\item 一大类量子霍尔效应(Quantum Hall)同源的物理系统和过程\cite{topo2016zoller,harper1,harper2,zak-expr-2013,chern-expr-2015,ab-expr-2015,wilsonline-expr-2016,haldane-expr-2014,charge-pump-expr-2016-de,charge-pump-expr-2016-jp,4dqhall-expr-2018},拓扑物态,(非交换)几何物理,如各种非平庸的 Berry 相\cite{ab-expr-2015},Zak 相\cite{zak-expr-2013},非阿贝尔 Wilson 路径\cite{wilsonline-expr-2016}等等,以及 Haldane 模型\cite{haldane-expr-2014},Harper-Hofstadter 模型\cite{harper1,harper2},Thouless 拓扑电子泵\cite{charge-pump-expr-2016-de,charge-pump-expr-2016-jp},四维量子霍尔效应\cite{4dqhall-expr-2018},等等。

\item 各种自旋模型\cite{spin-chain-expr-2011},包括 Ising 模型,Heisenberg 模型,XXZ 模型等\cite{olbook}。

\item 无序系统\cite{mbl1d,mbl2d},包括 Anderson 局域化和多体局域化等系统,利用不公度的晶格实现准无序系统的 准一维 Aubry-Andre 模型\cite{mbl1d},二维多体局域化模型\cite{mbl2d}等。

\item 不同维度的物理,包括一维\cite{incommensurate,spin-chain-expr-2011,af2,hidden_af_doped} Luttinger 液体,自旋电荷分离,非公度的自旋波和电荷波激发\cite{incommensurate},二维的短程关联和准长程序\cite{microscope5,microscope6,af1,af3,canted,af_long_range,pair_attractive,twobody-2017}等等。

\end{itemize}

% 等等。

此外,利用超冷原子光晶格体系还可以实现一些传统凝聚态很难实现的系统与过程,如周期含时驱动的系统和动力学淬灭过程。一方面,周期含时驱动的体系给人们 提供了实现一些有效稳态模型的方式,例如通过高频率周期含时驱动所生成的 Floquet-拓扑绝缘体\cite{floq-ti-2011},和前面提到的 Haldane 模型的实现\cite{haldane-expr-2014}与关联隧穿 Hubbard 体系\cite{floq-hubb-expr-2018,correlated-tunnel-expr-2018-shaking};另一方面,与静态系统相比,周期含时驱动本身就会使系统在诸如拓扑性质等内在性质方面有很大区别,例如,对周期驱动体系的拓扑分类与静态系统有着内禀的不同\cite{floq-edgestate-2013-prx}。此外,通过对这些系统的研究,人们可以研究一些传统凝聚态很少研究或很难用实验研究的物理过程,例如非平衡态的问题,还有从单个纯态出发的动力学淬灭的过程,甚至包括本征态热化假说(Eigenstate Themalization Hypothesis)\cite{thermalize-entropy-2016}这样的量子力学基本问题等。




\section{光晶格系统中的拓扑物态}


凝聚态物理中一个重要的主题是对不同的物质的相的刻画与分类。传统的朗道相变理论从对称性出发来描述物质的相和相变,即物质的相由体系的对称性以及态的对称性自发破缺来刻画,态的对称性自发破缺形成物质的序。然而,从上世纪80年代开始,一系列对量子霍尔效应的研究打开了新世界的大门,有一类物态并没有破缺体系的对称性,但却可以定义某种序,这些序决定着体系最基本的一些性质(例如材料的横向电导),而且对于材料的参数变化具有鲁棒性。这样的序可以用某个拓扑不变量来刻画,被称为体系的拓扑数,这类物态被称为拓扑物态。体系的某些基本性质,例如量子化的霍尔电导平台,都由这个拓扑数来刻画。材料的连续的微小的扰动不会改变体系的拓扑数,相应的也不会改变体系的那些基本的性质,只有发生较大变化,使体系经历量子相变(关闭能隙再打开),态的拓扑数才有可能改变。

拓扑物态中有一类拓扑绝缘体在过去十几年中被广泛研究\cite{topo2010hasan}。拓扑绝缘体是指一类电子材料,它们的体材料如同普通的能带绝缘体一样,电子填充价带而导带空置,价带和导带之间有能隙,因此体材料并不导电,但其边界或表面上却局域着导电的边缘态。这种边缘态受材料的拓扑数所保护,对于材料的微小、连续的扰动具有鲁棒性,并且对于体系的微小的无序性具有鲁棒性,不会与其他态发生背散射,这也正是其具有量子化的霍尔电导平台的原因。

拓扑物态所具有的良好性质吸引人们进行了许多理论和实验的研究,包括对拓扑材料的探索。而拓扑绝缘体之能发生于周期性晶格结构,也使得利用超冷原子光晶格系统进行探索成为可能。然而超冷原子系统进行研究的一个问题是,如何模拟电磁规范场。由于光晶格中的中性原子并不带电,并不能通过直接加电场或磁场的方式来使体系产生电磁规范场。对此,人们可以用光诱导产生人工规范场\cite{lightgauge2014}来进行研究。通过拉曼光耦合原子不同的内态,可以实现有效的自旋轨道耦合。类似地,通过拉曼光的帮助,利用Aharonov-Bohm 效应(以下简称AB效应)的原理,可以实现晶格方块上的磁通\cite{harper1,harper2}。这样的模型模拟了凝聚态物理中著名的 Hofstadter-Harper 模型\cite{topobook},在某些系统参数区间内,这样的模型具有非平庸的拓扑态\cite{topobook}。此外,人们还可以对光晶格系统进行周期性含时驱动,使频闪意义上的有效静态模型具有非平庸的拓扑。用这样的方式,人们在冷原子系统中实现了 Haldane 模型\cite{haldane-expr-2014},一个典型的 Chern 类绝缘体。通过含时驱动来生成具有非平庸拓扑相的有效静态模型,类似的方案还有前面提到的 Floquet-拓扑绝缘体\cite{floq-ti-2011},以及对光晶格的相位进行周期驱动来做到\cite{zhengwei-floquet-2014}。

近十年来,人们在利用激光对超冷原子气体系统进行调控的技术方面取得的进展,使得超冷原子气体系统作为一个量子模拟的平台可以很好的进行许多量子霍尔效应同源的物理系统和过程的模拟与研究\cite{topo2016zoller}。例如,德国马普所的 Immanuel Bloch 教授领导的实验组从2012年起陆续实现了从最简单但非平庸的一维的例子开始许多拓扑非平庸的物理系统和过程。2013年,他们报告了对一维 Su-Schrieffer-Heeger 模型(以下简称 SSH模型)\cite{ssh1979}中的非平庸的 Zak 相角\cite{zak1989}进行观测的实验\cite{zak-expr-2013};2015年他们报告了对 Hofstadter 模型中的 Chern 数\cite{chern-expr-2015}进行观测和测量的实验结果;同年,他们报告了对蜂巢晶格能带中的 Dirac 点进行 AB 效应的干涉实验\cite{ab-expr-2015};次年,他们报告了基于上一个实验进行的更进一步的 Wilson 路径的研究\cite{wilsonline-expr-2016},研究显示,不止拓扑性质是重要的,参数空间中的几何结构也是重要、而且可测量的,这样的实验研究对于进一步进行非阿贝尔过程的实验研究,甚至分数量子霍尔效应中的任意子激发等的研究都是重要的铺垫;2018年,他们报告了对四维量子霍尔效应基于电荷输运的研究\cite{4dqhall-expr-2018}。

同时,世界上还有许多其他的冷原子实验小组也在进行拓扑物理的研究。例如苏黎世联邦理工学院的 Tilman Esslinger 教授领导的实验小组,在2014年报告了他们利用周期驱动光晶格实现 Haldane 类型的有效静态模型的研究\cite{haldane-expr-2014},他们的实验中观测到了清晰 Haldane 拓扑相图,在超冷费米原子气体系统第一次直接证实了 F. D. M. Haldane 在1988 年的提出的量子反常霍尔效应的有效模型\cite{haldane1988}。该实验中利用到了圆偏晃动的方式来周期性驱动光晶格\cite{oka2009},也展示了周期驱动光晶格系统的强大威力。

还有一些拓扑物态相关的现象和过程,如 Thouless 式的量子化绝热电子输运\cite{thouless1983},也在超冷原子光晶格系统中得到实现\cite{charge-pump-expr-2016-de,charge-pump-expr-2016-jp}。人们对 Rice-Mele 型\cite{ricemele1982}的拓扑电子输运进行了直接的观测\cite{charge-pump-expr-2016-de,charge-pump-expr-2016-jp},并对原子云的中心位置进行直接测量,得到了很好的量子化的结果,验证了 Thouless 的理论;人们还通过调节冷原子系统超晶格的叠加方式来实现 Hofstadter模型的电子输运\cite{charge-pump-expr-2016-de},并以此来匹配不同参数下的拓扑相\cite{charge-pump-expr-2016-de}。此外,人们还实现了玻色子的 Hofstadter-Harper 模型\cite{twobody-2017},并对其中的少体相互作用进行了研究。

以上种种研究展示了超冷原子气体光晶格系统对拓扑物态相关的一系列研究,其调控细致、精密,测量手段丰富、差异化,测量结果干净。总而言之,调控手段丰富,系统干净可控,可以模拟的模型类型和参数区间巨大,是一个很好的研究拓扑物态以及一系列量子霍尔效应同源的物理系统和过程的平台。高度的可控性和可测性使人们可以专注于研究所关注的物理,那么在这方面作出深入细致的理论研究也就具有重要价值,例如,提出非平庸的模型、过程,和在超冷原子光晶格系统中可实现的方案\cite{creutz}。(参见第 \ref{sec:creutz} 节)



\section{光晶格系统的动力学性质}

超冷原子系统的一个重要特点是高度的可调控性。高度的可调控性带来的是可模拟的物理系统和过程的丰富性。超冷原子系统的另一个重要特点是测量手段的多样性和差异化。而测量手段的多样使得对所研究对象的探测可以从不同角度进行。近年来超冷原子实验系统在测量技术方面有许多重要进展,例如费米子量子显微镜\cite{microscope1,microscope2,microscope3,microscope4,microscope5,microscope6}的实现,使得人们可以对原子气体在实空间中的分布进行直接的探测,并且做到时间分辨(time-resolved)的测量。这也使得对光晶格含时动力学的研究成为可能。

对含时动力学的研究有别于传统凝聚态中关注平衡态性质的研究。例如,一个可以被 Hubbard 模型所描述的体系中相互作用的行为是排斥还是吸引可以使体系的平衡态性质有本质的、定性的不同,例如对于二维正方晶格上半填充的体系,强排斥相互作用的体系基态倾向于形成反铁磁序,而强吸引相互作用的体系则倾向于形成超导Cooper配对激发和电荷密度波\cite{nagaosa}。然而,两种平衡态时截然不同的体系在动力学演化方面却可能有惊人的相似性(见实验\inlinecite{hubbard-expan-2010,hubbard-expan-2012},和第 \ref{sec:dynsymm} 节),这揭示了它们其中的某种联系。超冷原子系统为这种动力学性质的研究提供了良好的平台。

过去,人们可以对冷原子系统可以做的典型测量包括时间飞行测量,通过对释放前后的冷原子气体云大小和位置的记录来近似计算其动量空间的分布。后来人们可以做到对超冷原子气体云实空间的分布做直接的测量,以此可以进行动力学过程的直接测量,例如前面提到的 Fermi-Hubbard 模型中的反常扩散现象\cite{hubbard-expan-2010,hubbard-expan-2012}。近来,得益于冷原子显微镜技术的进步,人们可以做到单粒子解析、单格点解析、时间分辨的测量,这使得人们可以对光晶格中的一些动力学过程进行深入单个粒子层面的、以及少体极限下的研究与探测。例如,2017年,Harvard 大学的 Marcus Greiner 教授领导的实验小组报告了他们对 相互作用的 Harper-Hofstadter 模型中玻色子两体相互作用极限下的显微观察与探测\cite{twobody-2017}。观测表明,粒子的两体运动具有手征特性,而且与相互作用密切相关。相互作用为吸引还是排斥,以及磁通的穿入还是穿出,两体运动都显示出一些对称性。这揭示了体系不同参数区间下的动力学过程中可能存在着受保护的对称性与精确关系,而与模型平衡态下基态性质的截然不同形成反差。此外,人们在对平衡态性质的研究中也显露出一些动力学相关的性质,例如在一维多体局域化的实验研究中\cite{mbl1d},实验所测量的物理量,奇偶格点的非平衡算符的期望值,也显示出了某种对于相互作用的对称性。种种这些,都启发我们对超冷原子光晶格体系的动力学过程进行定性与定量的理论分析。

受这些实验现象的启发,我们发现了类似这样的 Hubbard 系统中动力学过程的受保护的对称性并证明了相关的定理\cite{dynsymm},详见第 \ref{sec:dynsymm} 节。这使得我们从理论层面对验中显示出的动力学过程的有趣现象进行理解,并得以将看似截然不同的实验与动力学过程统一起来。该定理也给出了类似系统中满足相应条件的普适性的关系。

对于动力学的研究不止局限于对某个特定态的动力学演化的研究,在人们对于平衡态或准平衡态的研究,或传统凝聚态类型的输运性质的研究中,也可以得到线索和启发。例如,2018年,Princeton 大学 Waseem S. Bakr 教授领导的实验小组和 MIT 的 Martin W. Zwierlein 教授领导的实验小组分别报告了他们对于 Fermi-Hubbard 模型中 电荷 输运性质 和 自旋 输运性质的研究,实验中他们分别对 电荷 扩散系数和 自旋 扩散系数进行了测量,并通过一些传统凝聚态方法,如 Boltzmann 方程的方法,进行了一定分析。而事实上,这两种看似截然不同的输运过程却因为某些体系的动力学对称性而产生联系。我们发现在某些特定条件下 Fermi-Hubbard 体系中电荷密度波与自旋密度波的动力学演化可以相互映射的动力学性质,并证明了相关的定理\cite{diffusion},详见第 \ref{sec:diffusion} 节。这样的动力学对称性决定了体系(准)平衡态的某些输运性质。

此外,还有一些关注淬灭过程的动力学研究,这里不再赘述。



\section{周期含时驱动的光晶格系统}

一类很有意思的系统是周期含时驱动的系统,这样的系统过去很难在传统的凝聚态体系中实现,但在光学系统中却很常见。周期含时驱动的系统可以看成广义的动力学研究的一部分,而它们
有一套很好的形式化的理论来描述,也就是 Floquet 理论\cite{floquet2017}。

Floquet 理论对于周期驱动系统给出了形式化的描述,并且对于高频驱动可以给出良好的解析近似结果\cite{highfreq2015},其方法本质上是微扰论,也称 Floquet-Magnus 展开\cite{floquet-magnus-2001}。由于含时驱动的周期性,体系在时间方向呈现出周期性,能谱也呈现出周期性的结构。能量本身不在是定义良好的量子数标签,此时好的量子数是准能量(quasi-energy),也就是能量模掉整数个驱动频率为单位的能量($\hbar\omega$),即
\begin{align}
\epsilon = E \mod m\hbar\omega
\end{align}
这与空间周期性的晶格体系在动量方向呈现出周期性结构,在数学上很类似,这种情况下,动量本身也不是一个定义良好的量子数标签,而应该是体系的准动量,也就是动量模掉整数个倒格矢($\vect{G}$),
\begin{align}
\vect{q} = \vect{k} \mod m \vect{G}
\end{align}

过去对 Floquet 研究较多的领域是光学领域\cite{shirley1965,sambe1973}。近年来,一些凝聚态多体系统中也实现了一些Floquet系统,如利用 Floquet 光脉冲对材料进行激发,形成光子激发的Floquet-拓扑绝缘体\cite{photon-floq-2013},以及拓扑绝缘体表面的 Floquet-Bloch 边界态\cite{floq-bloch-2013}。但总体而言,在固体材料中进行周期性驱动产生类似的 Floquet 体系仍然很少。

超冷原子气体系统是实现 Floquet 多体体系的另一个平台。利用超冷原子光晶格系统,并对系统进行周期性含时驱动,人们得以产生出周期性含时的多体系统并对之进行相干的调控\cite{floquet2017}。对光晶格的驱动频率在几百至千赫兹量级,相应的能量量级正是与光晶格低能带与高能带的能带间隙可比的能量量级。这样的对超冷原子多体系统进行的周期性相干驱动可以产生出许多非平庸的有效模型和新奇的物理过程。

\begin{itemize}

\item 有一些具有非平庸的拓扑结构,例如前面提到的 Haldane 模型的冷原子实验实现\cite{haldane-expr-2014},即是利用了对光晶格进行圆偏的晃动来实现带有非平庸拓扑的有效频闪模型\cite{oka2009},类似的还有利用周期驱动实现Floquet拓扑绝缘体的方案\cite{floq-ti-2011},和利用晃动光晶格实现非平庸拓扑模型和边界态的方案\cite{zhengwei-floquet-2014}。

\item 还有一些则具可以产生具有强关联效应的模型,产生出如关联隧穿效应的存在\cite{correlated-tunnel-expr-2018-shaking,correlated-tunnel-expr-2018-raman},进而可以存在新奇的量子相结构和相变过程\cite{floqhubb}。这样的有效模型有助于理解高温超导赝能隙等物理。

\item 特别的,周期驱动本身可以使系统的幺正演化具有非平庸的性质,例如不同于一般静态模型的拓扑分类\cite{floq-edgestate-2013-prx}。在2013年 Mark S. Rudner 等人的工作中\inlinecite{floq-edgestate-2013-prx},周期含时系统的拓扑分类由于准能谱的存在,具有与静态体系迥然不同的拓扑分类性质。准能谱的“上不封顶下不见底”使基带不再是一个良好的定义,进而一些对于静态模型的拓扑分类\cite{topoclassify2016}的关系和结论不再适用;某些参数下呈现拓扑平庸的静态模型却有着非平庸的Floquet 对应,而这也体现在体系在开放的边界上出现的局域化的手征边界态上。

\end{itemize}

值得注意的是,这些新奇的物态与过程本质上都是由对多体量子系统的周期性相干驱动所诱导的。这样的过程在传统的凝聚态材料中较难实现,但超冷原子气体系统特别是光晶格系统为之提供了很好的实现平台。作者博士期间深入学习了 Floquet 的形式理论(详见第 \ref{sec:floq:theory} 节),并且针对由周期含时驱动的关联隧穿效应作出了相关的理论研究\cite{floqhubb}(详见第 \ref{sec:floqhubb} 节)。作者和合作者们发现,在这样的体系中可以存在新奇的铁磁相和相分离结构,其诱因完全由近共振的周期驱动引起,有别于传统凝聚态体系中众所周知的 Stoner 机制(详见第 \ref{sec:floqhubb} 节)。





\section{量子多体物理与机器学习}

机器学习是一门古老的学科。早在二百年前,勒让德、高斯就发明了最小二乘法(Least Squares),最小二乘法被高斯用来拟合天体的运行轨迹。一百年前,人们就发明了主成分分析法(Principal Component Analysis,PCA)\cite{pearson1901,hotelling1933},并在农业生产中加以利用\cite{pcabook}。上世纪50年代,人工智能的概念诞生,在随后的半个多世纪里,人工智能相关技术的发展及应用经历了多次高潮和低谷。而作为其中核心的机器学习算法,也在半个多世纪的时间里发展壮大,并逐渐系统化,诞生了诸如玻尔兹曼机、人工神经网络等概念与算法。然而,受限于计算机的算力与数据的缺乏,在很长一段时间里机器学习算法并不能发挥出其巨大作用。近年来,受益于算法的改进与算力的提高,以及大数据的积累,人造智能尤其是机器学习算法在许多方面呈现出爆发式的增长。其中,以深度学习\cite{dl2015}为代表的一系列算法在语音识别、语义翻译、机器视觉等方面取得了巨大的成功,在特定任务上的识别精准度已经超越人类。同时,人工智能产业的发展也初具规模,机器学习技术在金融、安防、交运、电子商务等领域得到初步应用,例如金融防欺诈、智能安防系统、无人驾驶、推荐系统等,都利用到了机器学习的技术。


2012年,主要由人造神经网络构成的 ImageNet 在图像识别的比赛中取得突破性成功\cite{imagenet2012}。2016年初,由 Google DeepMind 开发的 AlphaGo 在与人类顶级棋手进行的围棋对决中以巨大优势战胜人类顶级围棋选手,这使人工智能与机器学习的概念再度成为公众焦点。而机器学习算法在模式识别与强化学习等方面的巨大成功也引起了学术界的广泛兴趣,一方面,有许多学科的科学家们希望能运用机器学习技术,为解决自己学科已有的问题提供更强大的工具,另一方面,机器学习本身的黑匣子也是人们探究的课题。与机器学习相关的交叉学科研究作为新兴的领域方兴未艾,蓬勃发展\cite{mlrev1903}。

事实上,早在机器学习火爆之前,科学家们就已经在科研中运用机器学习的技术。例如,利用机器学习技术进行数据分析与处理,帮助寻找基本粒子,如希格斯子,帮助进行天文观测,以及帮助进行肿瘤的分类等。近年来,在机器学习与基础物理学,如凝聚态物理与冷原子物理,相结合的方面的研究日益增多。传统的凝聚态多体系统具有$10^{23}$以上量级的自由度,对于数值模拟是巨大的挑战。而机器学习算法在数据降维等方面具有优势。人们利用机器学习算法进行量子多体数值计算方法的优化等,也试图从交叉学科的角度理解机器学习的黑匣子。此外还有机器学习与量子计算相结合的研究\cite{qml2017,qai2017},两个领域相辅相成。

机器学习算法按学习的类型可大致分为\cite{prmlbook}非监督式学习(Unsupervised Learning)和监督式学习(Supervised Learning),还有强化学习(Reinforcement Learning)。非监督式学习包括聚类算法(Clustering)、主成分分析法等,监督式学习包括各种回归算法(Regression)、分类算法(Classification)、支持向量机(Supportive Vector Machine)、决策树和随机森林(Decision Trees and Random Forests)等,强化学习包括主动学习(Active Learning)等。这些算法中的许多都已经在一些研究中有所体现。迄今,人们在量子多体物理与机器学习相关领域进行的研究有如下几类。

\begin{itemize}

\item 量子多体系统数值算法改进与加速,如利用脊回归加速量子蒙特卡罗算法(Quantum Monte Carlo)\cite{acmc1,acmc2,acmc3,acmc4},利用推荐系统受限玻尔兹曼机(Restricted Boltzmann Machine,RBM)加速量子蒙卡\cite{acmcwl1,acmcwl2}等。

\item 机器学习识别凝聚态多体系统的相图、序参量,相变的探测与锚定,相分类,临界指数的提取等问题\cite{ml-anderson-2014,confusion,mlphase2017-nphys,wangleipca2016,wcpca,mlphase2017-prx,jp1,jp2,jp3,jp4,zhangyiml2017,zpf2017,topoml,wanxin-2017,kernel2017,ml-mbl-2017,pca2017a,pca2017b,rbm-2017,unsup-2017,unsup-hubb-2018,discriminative2018,ml-disorder-2018}。其中包括了运用非监督式学习方法\cite{unsup-2017,unsup-hubb-2018},如主成分分析法\cite{wangleipca2016,wcpca,pca2017a,pca2017b}等,运用监督式学习方法,如全连接神经网络\cite{zhangyiml2017,mlphase2017-prx,ml-mbl-2017}、卷积神经网络\cite{zpf2017,jp1,jp2,jp3,jp4,mlphase2017-nphys,wanxin-2017}、判别式神经网络\cite{discriminative2018}等,运用监督式学习与非监督式学习相结合的方法,如 Evert P. L. van Nieuwenburg 等人在2017年发表的工作\cite{confusion}中创立的“混淆”网络(Confusion),还有诸如受限玻尔兹曼机\cite{rbm-2017}、支持向量机与核方法\cite{kernel2017}等算法。这其中又涵盖了从无序系统\cite{ml-disorder-2018},如 Anderson 局域化\cite{ml-anderson-2014,jp1,jp2,jp4}、多体局域化\cite{ml-mbl-2017}系统,到 Bose-Hubbard 和 Fermi-Hubbard 模型等玻色子和费米子模型\cite{confusion,rbm-2017,mlphase2017-prx,pca2017b,unsup-hubb-2018},Ising 模型\cite{jp3,kernel2017,mlphase2017-nphys,wangleipca2016,wcpca,wanxin-2017}、各种晶格上的Heisenberg 模型、XY模型\cite{pca2017a,unsup-2017,wcpca}等自旋模型等一系列传统凝聚态强关联系统的相图和相变研究,以及到拓扑绝缘体的拓扑相分类\cite{zhangyiml2017,zpf2017,jp1,jp2}等。

\item 机器学习量子理论,如量子力学薛定谔方程\cite{dlshrodinger2017,wyd2018}、量子多体形式理论\cite{ml-manybody}等。

\item 理解机器学习算法的黑匣子\cite{mlanalysis2017lin},运用物理学的方法与概念,如重整化群\cite{dlrg2013,maprgdl2014,rgml2018}、张量网络、全息对偶原理\cite{yyz2018}等,结合机器学习算法的研究。

\item 机器学习帮助设计量子实验\cite{activelearn-exprdesign-2018},制备量子态与进行量子控制\cite{rein-2018-prx}等。

\item 量子机器学习算法的研究\cite{qml2017,qai2017},如量子主成分分析\cite{qpca2014},量子主动学习机器人\cite{active-agent-2014},量子增强机器学习\cite{qeml}等。

\item 机器学习与自动化材料接口,如利用机器学习算法自动化学习材料的密度泛函\cite{wcyyz}等。

\end{itemize}


机器学习本身为数据处理与分析、数值算法优化、自动化模式识别等任务提供了强大的算法工具。在利用这些已有算法工具的同时,如何创造性地解析机器学习的算法原理,并使之具有推广能力,对于理解机器学习的本质,与未来实现从弱人工智能的目标到强人工智能的愿景来说具有重要价值。在第 \ref{sec:topoml} 章节中,我们将介绍我们在利用深度学习人造神经网络进行拓扑物态分类学习方面的工作\cite{topoml},在此工作中,我们一定程度上理解了机器学习“学会”的方式,并将神经网络训练得上具有一定的推广能力(能够识别没有见过的拓扑类)——这也是目前,人,比起机器的优势所在。


